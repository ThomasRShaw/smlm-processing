\documentclass{article}

\usepackage{amsmath,amssymb}

\usepackage{graphicx}

\title{Documentation for STORManalysis}
\author{Thomas Shaw}

\begin{document}
\maketitle


\section{Drift Correction}
Stage drift over the course of a localization-microscopy experiment can be
substantial, and correcting is important to obtaining optimal measurement
accuracy.

\subsection{Algorithms}

All of the algorithms available in this package rely on the assumption that
emitters and/or larger structures are fixed in place, but are detected many
times over the course of an experiment. Thus, a spatial cross-correlation of
localizations from two different time intervals should have a peak centered
at a displacement of 0. Normally, there will be a primary peak due to multiple
localizations of the same emitter (with width given by the localization precision).
Secondary peaks, also centered at 0, may also be present, due to larger structures
such as whole cells. The width of these peaks will depend on the length-scales of
the structures.

If there has been drift between the two time-intervals, the cross-correlation peak
will no longer be at 0 displacement, but will correspond to the drift.

\subsection{parameters to specify}



\end{document}
